\documentclass[10pt]{jarticle}
\usepackage{float}
\usepackage{adrobo_abst}
\usepackage[dvipdfmx]{graphicx}
\usepackage{amssymb,amsmath}
\usepackage{bm}
\usepackage[superscript]{cite}
\usepackage{enumerate}
\usepackage{url}
\usepackage{booktabs}
%\usepackage[absolute]{textpos}

\renewcommand\citeform[1]{(#1)}

\begin{document}
    
    \makeatletter
    \doctype{2025年度卒業論文概要}
    \title{色彩反転された食品がもたらす食欲減退効果の検証}{}
    \etitle{Verification of Appetite Suppression Effects Induced by Color Inversion}{}
    
    \author{22C1070\hspace{.5zw}鈴木航}
    \eauthor{Ko SUZUKI}
    
    \makeatother
    
    \abstract{This research explores the impact of color inversion of foods on appetite to prevent overeating. 
    We developed an application for head-mounted displays that performs real-time negative-positive reversal of food appearance. 
    The experimental results suggested that color-inverted foods elicited significant psychological reactions, 
    such as reduced appetite and diminished taste perception.
    }

    
    \keywords{Appetite Suppression, Color Inversion, Visual Manipulation, Head-Mounted Display, Eating Behavior}
    
    \maketitle
    
    \supervisor{指導教員:上田 隆一 教授}
    
    \section{緒\hspace{2zw}言}%==========================
    近年,食習慣の変化に伴う肥満や生活習慣病の増加が深刻な社会問題となっている.食べ過ぎを防ぐ方法として食品の色彩を操作し食欲を抑制するアプローチがある.
    先行研究\cite{奥田ら2002}では,食品の色彩は「美味しさの評価」や「摂食意欲」と密接に関係しており,暖色系は食欲を促進し,寒色系はそれを減退させることが報告されている.

    食品の色彩を変化させる手法として,AR(Augmented Reality:拡張現実)技術の活用がある.AR技術を用いれば,装着型ディスプレイであるHMD(Head Mounted Display)などを介して,
    現実の食品の質感を維持したまま色彩のみをリアルタイムで操作することが可能となる.

    村井\cite{村井ら2020}の研究では,シースルー型HMDを用いて食品に青色フィルタを重畳させることで,美味しく食べながらも満腹感を高め食事量を抑制する手法が提案されている.
    しかし,特定の色を重畳する方法では,視覚的変化が色味の違いに限定される.

    特定の色の重畳よりも劇的な視覚変容をもたらす手法として,ネガポジ反転(補色変換)がある.
    ネガポジ反転処理は,色味だけでなく明度や彩度を網羅的に逆転させるため,図1のような強い違和感を創出できる.

    そこで本研究では,特定の色の重畳よりも劇的な視覚変容をもたらす処理としてネガポジ反転を行い,
    喫食者の食欲,満足度,および味覚知覚にどのような影響を与えるかを実験的に明らかにする.

    \begin{figure}[h]
    \begin{center}
      \includegraphics[width=0.2\linewidth]{fig/negapozi_noodle.jpg}
      \caption{Color-inverted ramen}
      \label{fig:exp_negapozi}
    \end{center}
    \end{figure}

    \section{ネガポジ反転システムの開発}
    ネガポジ反転処理を実現するため, VRヘッドマウントディスプレイである「Meta Quest 3」を用いたシステムを開発した.
    Meta社が提供しているUnityプロジェクト「Passthrough Camera Api Samples」を基に, 
    カメラで映し出された映像の指定範囲のピクセルに対してネガポジ反転処理を行うアプリを実装した. 
    このアプリでは, カメラから入力されたRGBの各輝度値 $I$ に対し, 出力値 $I'$ を $I' = 1.0 - I$ と定義することで, 色相と明度を反転させている. 

    \begin{figure}[H]
     \begin{center}
      \includegraphics[width=0.75\linewidth]{fig/negapozisystem.png}
       \caption{Color inversion system}
        \label{fig:exp_system}
     \end{center}
    \end{figure}

    \section{実\hspace{2zw}験}
    被験者は20から26歳の男性10名とし,図2のようにMeta Quest 3を装着した状態で
    市販の同一製品のカップラーメンをネガポジ反転処理された状態と通常の視界で摂取した.
    また被験者には食事前と食事後にそれぞれアンケートを行い,見た目の魅力, 食欲, 味の知覚, 不快感, 満足度等の6項目(7段階リッカート尺度)ついて調査した.

    本実験では, 提示順序による順序効果(慣れや満腹感の影響)を相殺するため, 被験者をランダムに以下の2群に割り当てた. 
    \begin{itemize}
    \item \textbf{A群}:1回目に「通常視界」, 2回目に「ネガポジ反転視界」の順で摂取
    \item \textbf{B群}:1回目に「ネガポジ反転視界」, 2回目に「通常視界」の順で摂取
    \end{itemize}

\begin{figure}[H]
  \centering
  \begin{minipage}[t]{0.48\linewidth}
    \centering
    \includegraphics[width=\linewidth]{fig/IMG_5322.jpg}
    \caption{Eating scene}
  \end{minipage} 
  \hfill
  \begin{minipage}[t]{0.5\linewidth}
    \centering
    \includegraphics[width=\linewidth]{fig/cupnoodle.jpg}
    \caption{Noodle observed\\ through HMD}
  \end{minipage}
\end{figure}


    実験結果を表1,2に示す.「見た目がおいしそうだった」という項目において, 通常時の平均値が5.3であったのに対し, ネガポジ視界時では1.7と大幅に低下した. 
    これに伴い, 「食べたい気持ちは続いたか」という項目も, 通常時の6.2からネガポジ時の2.9へと大きく減少している. 
    このことから, 食品のネガポジ化は, 対象物の食欲を著しく減退させ, 食意欲の維持を困難にさせることが示唆された.

    「味を通常通り感じたか」という項目では, 通常時6.2に比べてネガポジ時3.6は低い評価となった. 食品自体は同一であるにもかかわらず, 
    視覚的な違和感が味覚の知覚や評価に対して負の影響を与えている可能性が示された. 自由記述においても, 
    「本来の味がわからなくなった」「色が不自然で脳が味を拒絶している感覚があった」といった意見が散見された.

    視界の状態に対する不快感は, 通常時の3.3に対し, ネガポジ時では3.8と微増した. しかし, 全体の満足度については通常時の5.7からネガポジ時の3.0へと大きく低下しており,
     単なる視界の不快さ以上に, 視覚情報の不自然さが食事体験全体の質を著しく阻害する要因となっていることが確認された.


\begin{table}[H]
  \centering
  \caption{Questionnaire results under normal visual conditions ($n=10$)}
  \label{tab:normal_results}
  \resizebox{\columnwidth}{!}{
    \begin{tabular}{lcccc}
      \toprule
      Item & Mean & Median & Standard Deviation \\
      Hunger level & 4.8 & 5.0 & 1.25 \\
      Desire to eat cup noodles & 4.9 & 4.5 & 1.04 \\
      Looked appetizing & \textbf{5.3} & 6.0 & 1.62 \\
      Perceived taste as usual & \textbf{6.2} & 7.0 & 1.25 \\
      Desire to continue eating & \textbf{6.2} & 7.0 & 1.17 \\
      Visual condition was uncomfortable & 3.3 & 3.0 & 1.95 \\
      Overall satisfaction & \textbf{5.7} & 6.5 & 1.68 \\
      \bottomrule
    \end{tabular}
  }
\end{table}

% --- 表2: ネガポジ時 ---
\begin{table}[H]
  \centering
  \caption{Questionnaire results under negative positive visual conditions ($n=10$)}
  \label{tab:negaposi_results}
  \resizebox{\columnwidth}{!}{
    \begin{tabular}{lcccc}
      \toprule
      Item & Mean & Median & Standard Deviation \\
      Hunger level & 4.7 & 5.0 & 1.85 \\
      Desire to eat cup noodles & 4.3 & 4.5 & 2.15 \\
      Looked appetizing & \textbf{1.7} & 1.0 & 1.19 \\
      Perceived taste as usual & \textbf{3.6} & 3.0 & 1.80 \\
      Desire to continue eating & \textbf{2.9} & 2.5 & 1.30 \\
      Visual condition was uncomfortable & 3.8 & 4.5 & 1.89 \\
      Overall satisfaction & \textbf{3.0} & 3.0 & 0.89 \\
      \bottomrule
    \end{tabular}
  }
\end{table}


    \section{結\hspace{2zw}言}%===========================
    本研究では,食べ過ぎを防ぐことを目的として,AR技術を用いた食品のネガポジ反転提示が喫食者の食行動に与える影響を検証した.
    実験の結果,ネガポジ反転処理は食品の「見た目のおいしさ」および「食欲の維持」を大幅に減退させることが確認できた.
    特に,食品そのものは同一であるにもかかわらず,味覚の正常な知覚が阻害され,「味が感じずらくなる」といった心理的反応が確認された.
    これは,色の重畳による既存手法以上に,ネガポジ反転による視覚的違和感が強力な摂食抑制因子として機能することを示唆している.
    一方で,長期的な利用における視覚的慣れの影響についてはさらなる検証が必要である.
    
    \vspace{5truemm}
    {\footnotesize
        \begin{thebibliography}{99}

            \bibitem{奥田ら2002}
            奥田 弘枝: ``Correlation between the Image of Food Colors and the Taste Sense'', 
            日本調理科学会誌, 
            Vol.35, (2002), pp.~2--9.


            
            \bibitem{村井ら2020}
            D. Murai, Y. Takegawa, A.Terai, K. Hirata: ``シースルー型HMDを用いた食べ物への動的な映像効果重畳による食欲減衰手法の提案'',
            研究報告エンタテインメントコンピューティング(EC),
            (2022)vol.30, pp.~1--3.

           

            
        \end{thebibliography}
    }
    \normalsize
    
\end{document}



%    \section{引用文献の書き方}%===========================
%    本文中の引用箇所には,右肩に小括弧をつけて,通し番号を付ける.例えば,文献\cite{工大2005}や,文献\cite{Shibutani2004, Handbook1979, Kikuchi2017, Adrobo2019}のようにする.

%    引用文献は,英文で記述されているもの(文献\cite{Shibutani2004}など)は英文で書き,本文末尾に引用順にまとめて書く.専門的な書籍(文献\cite{Handbook1979}など)についても引用しても良い.
%    Web上の資料を引用する場合,例えばオンラインジャーナルなどの場合は文献\cite{Kikuchi2017}のように,webページの場合は文献\cite{Adrobo2019}のように,それぞれ参考文献として記載して引用する.この時,URLとともに参照日を記載すること.ただし,webページの場合は個人の技術ブログなどのように第3者による十分な審査が行われていないものの引用は行ってはいけない.公的な機関が発行しているページであっても,その永続性の問題から必要最小限に留めることを推奨する.

%            \bibitem{Adrobo2019}
%            千葉工業大学 未来ロボティクス学科 学科概要: 
%            \url{http://www.robotics.it-chiba.ac.jp/ja/subject/index.html}, 
%            (参照日 2023年1月29日). 
    
%    \section{記号・単位の書き方}%===========================
%    \begin{table}[!h]
%        \begin{tabular}{lcl}
%            $L$ & : & 長さ [m] \\
%            $t$ & : & 時間 [s] \\
%            $x$ & : & 流れ方向の座標 [m] \\
%            $\alpha$ & : & 熱伝達率 [$\mathrm{W/(m^2\cdot K)}$] \\
%            $Re$ & : & レイノルズ数 \\
%            $\bm{R}$ & : & 回転行列 \\
%            $\bm{t}$ & : & 並進ベクトル \\
%        \end{tabular}
%    \end{table}
    
%    量記号はイタリック体,単位記号はローマン体,無次元数はイタリック体で書く.
%    数学記号・単位記号及び量記号は,半角英数字を使用する.単位は,SI単位を使用し,4~MPa のように書く.
    
%    分野によって作法は異なるが,わかりやすさの観点から行列の表記は大文字のイタリック体・ボールド体を推奨し,ベクトルの表記は小文字のイタリック体・ボールド体を推奨する.
    
%    \section{見出しの書き方}%===========================
    
%    論文の章立ては.章・節・項である.章見出しはゴシック体で記述し,2行分をとって行の中ほどに書く.
%    18字以上は3行分を必要とするが,見出しが不必要に長くなるのは推奨されない.
    
%    \subsection{節の書き方}
    
%    項の見出しもゴシック体で記述し,本文は見出し後に改行をせずに,直後に2文字分の空白を空けてから書き始める.
    
%    \begin{center}
%            \includegraphics[width=0.45\textwidth]{./fig/sample.png}
%            \caption{Sample of clear figure}
%            \label{fig:sample-fig}
%        \end{figure}
%   \end{center}
    
%    \section{図及び写真・表の作成に関して}%===========================
%    \begin{enumerate}
%        \setlength{\parskip}{0cm} % 段落間
%        \setlength{\itemsep}{0cm} % 項目間
%        \item 本文中では,\reffig{sample-fig},\reftab{sample-tab}のように日本語で書く.写真は,図として扱う.
%        \item 番号・説明(キャプション)などは,図・写真についてはその下に,表についてはその上に書く.
%        \item 本文と,図・表の間は1行以上の空白を空けて,見やすくする.
%        \item 図中・表中の説明及びキャプションはすべて英語で書く(最初の文字は大文字とする).
%        \item 図及び表がl列(片側)に収まらない場合2列(両側)にまたがって書くことができる. 
%        \item 図及び表の横に空白ができても,その空白部には本文を記入してはならない.
%    \end{enumerate}
    
%    \begin{table}[t]
%        \caption{Sample of expression of values}
%        \label{tab:sample-tab}
%        \begin{center}
%            \vskip 1zh
%            \begin{tabular}{|c|c|}
%                \hline
%                Recommend & Not recommend \\ \hline
%               $0.357$ & $.357$ \\ \hline
%                $3.141\ 6$ & $3.141,6$ \\ \hline
%                $3.141\ 6 \times 2.5$ & $3.141\ 6 \cdot 2.5$ \\ \hline
%            \end{tabular}
%       \end{center}
%    \end{table}
    
%    \section{数式の書き方}%===========================
    
%    式番号は,式と同じ行に右寄せして( )の中に書く.また,本文で式を引用するときは,\refeqn{sample-eq1}のように書く.
%    \begin{equation}
%        \gamma(t) = \frac{ji}{N} \label{eq:sample-eq1}
%    \end{equation}
%    \begin{equation}
%    \bar{C}(t)=\frac{1}{N}\sum_{i=1}^{N}C_i(t) \label{eq:sample-eq2}
%    \end{equation}
    
    %\begin{table}[!b] \notag
    %\begin{minipage}{\textwidth}
    %\begin{tabular*}{\textwidth}{@{\extracolsep{\fill}}lr}
    %{\footnotesize
    %$\displaystyle 
    %u^*_{i,j} = u^n_{i,j} - \Delta t \left\{u^n_{i,j}\frac{u^n_{i+1,j} - u^n_{i-1,j}}{2\Delta x} + v^n_{i,j}\frac{u^n_{i,j+1} - u^n_{i,j-1}}{2\Delta y} + \frac{1}{Re}\left(\frac{u^n_{i+1,j} - 2u^n_{i,j} + u^n_{i-1,j}}{(\Delta x)^2} + \frac{u^n_{i,j+1} - 2u^n_{i,j} + u^n_{i,j-1}}{(\Delta y)^2}\right) \right\}
    %$}
    %& $\inlineTag\label{eq:long-eq} $
    %\end{tabular*}
    %\end{minipage}
    %\end{table}
    
%    式を書くときは,2文字分空白を空ける.
%    また,必要行数分を必ず使うようにして書く.
%    3行必要とする式を2行につめて書いたり,2行に分かれる式を1行に収めたりしない.
%    なお,本文と式,式相互間は1行以上の空白を空けて,見やすくする.
%    ポイント数は本文に準じるものとするが,添え字等が小さく読みにくくなるときは適宜拡大する.
    
    %式はなるべく片側に書くことが望ましいが,両側にまたがる場合は,読む順序に混乱を生じないように,そのページの式の上,または下の本文全部を両方にまたがるように書かなければならない.
    %本見本では\refeqn{long-eq}のようにページの最上段もしくは最下段に配置している場合は,上記のような混乱は生じ得ないので以下の文章は2段組で続けることができる.
    %ただし,所望の位置に表示されない,文字が重なるなどレイアウト上の多くの問題が生じるため,極めて推奨しない.
    
